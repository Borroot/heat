\documentclass[a4paper]{article}

\author{Steven Bronsveld\\Wouter Damen\\Kirsten Hagenaars\\Bram Pulles}
\title{\textbf{Heat Diffusion: openMP and MPI}}

\begin{document}
\maketitle

\tableofcontents

\pagebreak
\section{Sequential}
The following possible improvements have been tried out in the sequential version of the algorithm.
\begin{itemize}
    \item Combining relaxation and stability check into one loop. Since the array resulting of the relaxation ends with some amount of zero's (possibly none) and we know where those zero's will start, we terminate the loop just before arriving at the zero's.
    \item Not combining relaxation and stability. Terminating the relaxation loop at the point for which we know there will only be zero's in the remainder of the \texttt{out} array, as explained in the previous bullet. Terminating the stability check once some \texttt{i} for which \texttt{(fabs(out[i] - in[i]) > eps)} holds is found, checking this condition is faster than keeping track of \texttt{res = res \&\& ( fabs( out[i] - in[i]) <= eps)}.
\end{itemize}
The following improvements have been made to the sequential version of the relaxation algorithm. This list includes those mentioned above that resulted in the fastest implementations.
\begin{itemize}
    \item Termination relaxation and stability loops as early as possible.
    \item Check for malloc failure.
    \item Free the allocated memory at the end of the program.
\end{itemize}
We made the last improvement after running valgrind on the program, which alerted us to the memory leak that was left by the heap-allocated vectors. 
N, EPS and HEAT can be passed as command-line arguments to the program for easy testing.

\section{openMP}

We found that openMP's reduction is not a good approach, because we want to stop searching once we find one instance of \texttt{fabs(out[i] - in[i]) <= eps} being false. Using openMP's support for for-loops in the \texttt{init} and \texttt{relaxAndStable} functions does achieve this.\\
Interestingly, though the "split" version performed faster in the sequential setting, the "joined" version performs faster in openMP. 
This might be because of the overhead caused by synchronizing the threads.

\section{MPI}
For MPI we compared the performance of versions in which  we use \texttt{MPI\_Allgather}
and \texttt{MPI\_Allreduce} and of versions in which we manually communicate the results using \texttt{MPI\_Send} and \texttt{MPI\_Recv}. It was very apparent that seperately using \texttt{MPI\_Allgather} for the heat array and \texttt{MPI\_Allreduce} for the stability boolean is much faster than manual messages.

\section{Performance}


\end{document}
